\documentclass[a4paper, 12pt]{report}

\usepackage[dvipsnames]{xcolor}

%%%%%%%%%%%%%%%%%
% Set Variables %
%%%%%%%%%%%%%%%%%

\def\useItalian{0}  % 1 = Italian, 0 = English

\def\courseName{Network Algorithms}

\def\coursePrerequisites{
    \begin{itemize}
        \item Progettazione di Algorithmi
    \end{itemize}
}

\def\book{TODO}

% \def\authorName{Simone Bianco}
% \def\email{bianco.simone@outlook.it}
% \def\github{https://github.com/Exyss/university-notes}
% \def\linkedin{https://www.linkedin.com/in/simone-bianco}

\def\authorName{Alessio Bandiera}
\def\email{alessio.bandiera02@gmail.com}
\def\github{https://github.com/aflaag-notes}
\def\linkedin{https://www.linkedin.com/in/alessio-bandiera-a53767223}

% Do not change

%%%%%%%%%%%%
% Packages %
%%%%%%%%%%%%

\usepackage{../../packages/Nyx/nyx-packages}
\usepackage{../../packages/Nyx/nyx-styles}
\usepackage{../../packages/Nyx/nyx-frames}
\usepackage{../../packages/Nyx/nyx-macros}
\usepackage{../../packages/Nyx/nyx-title}
\usepackage{../../packages/Nyx/nyx-intro}

%%%%%%%%%%%%%%
% Title-page %
%%%%%%%%%%%%%%

\logo{../../packages/Nyx/logo.png}

\ifx\useItalian0
    \institute{\curlyquotes{\hspace{0.25mm}Sapienza} Università di Roma}
    \faculty{Ingegneria dell'Informazione,\\Informatica e Statistica}
    \department{Dipartimento di Informatica}
    \subtitle{Appunti integrati con il libro \book}
    \author{\textit{Autore}\\\authorName}
\else
    \institute{\curlyquotes{\hspace{0.25mm}Sapienza} University of Rome}
    \faculty{Faculty of Information Engineering,\\Informatics and Statistics}
    \department{Department of Computer Science}
    \subtitle{Lecture notes integrated with the book \book}
    \author{\textit{Author}\\\authorName}
\fi

\title{\courseName}
\date{\today}

% \supervisor{Linus \textsc{Torvalds}}
% \context{Well, I was bored\ldots}

%%%%%%%%%%%%
% Document %
%%%%%%%%%%%%

\begin{document}
    \maketitle

    % The following style changes are valid only inside this scope 
    {
        \hypersetup{allcolors=black}
        \fancypagestyle{plain}{%
        \fancyhead{}        % clear all header fields
        \fancyfoot{}        % clear all header fields
        \fancyfoot[C]{\thepage}
        \renewcommand{\headrulewidth}{0pt}
        \renewcommand{\footrulewidth}{0pt}}

        \romantableofcontents
    }

    \introduction

    %%%%%%%%%%%%%%%%%%%%%

    \chapter{TODO}

    \section{TODO}
    
    \subsection{Classical solutions}
    
    \begin{algorithm}[H]
        \caption{
            \textit{Bellman-Ford}: TODO
        }

        \label{alg:bellman_ford}
        \begin{algorithmic}[1]
            \Function{bellmanFord}{$G$}
                \State TODO
            \EndFunction
        \end{algorithmic}
    \end{algorithm}

    \begin{algorithm}[H]
        \caption{
            \textit{Dijkstra}: TODO
        }

        \label{alg:dijkstra}
        \begin{algorithmic}[1]
            \Function{dijkstra}{$G$}
                \State TODO
            \EndFunction
        \end{algorithmic}
    \end{algorithm}

    \begin{algorithm}[H]
        \caption{
            \textit{Floyd-Warshall}: Given a directed graph $G$, and an unconstrained weight function $w$ for the edges, the algorithms returns a matrix \texttt{dist} such that $\arraytt{dist}{u}{v}$ is the weight of the least-cost path from $u$ to $v$.
        }

        \label{alg:floyd_warshall}
        \begin{algorithmic}[1]
            \Function{floydWarshall}{$G$, $w$}
                \State Let $\texttt{dist[}n\texttt{][}n\texttt{]}$ be an $n \times n$ matrix, initialized with every cell at $+ \infty$
                \For{$u \in V(G)$}
                    \State $\arraytt{dist}{u}{u} = 0$
                \EndFor
                \For{$(u, v) \in E(G)$}
                    \State $\arraytt{dist}{u}{v} = w(u, v)$
                \EndFor
                \For{$k \in V(G)$}
                    \For{$u \in V(G)$}
                        \For{$v \in V(G)$}
                            \State $\arraytt{dist}{u}{v} = \min \rbk{\arraytt{dist}{u}{k}, \arraytt{dist}{k}{v}}$
                        \EndFor
                    \EndFor
                \EndFor
            \EndFunction
        \end{algorithmic}
    \end{algorithm}

    \idea{
        The core concept of the algorithm is to construct a matrix using a \href{https://en.wikipedia.org/wiki/Dynamic_programming}{dynamic programming} approach, that evaluates all possible paths between every pair of vertices. Specifically, to determine the shortest path from a vertex $u$ to a vertex $v$, the algorithm considers two options: either traveling directly from $u$ to $v$, or passing through an intermediate vertex $k$, potentially improving the path.
    }

    \cost{
        The \texttt{for} loop in line 3 has cost $\Theta(n)$, the \texttt{for} loop in line 6 has cost $\Theta(m) = \Theta(n^2)$ and the cost of the triple nested \texttt{for} loop is simply $\Theta(n^3)$. Therefore, the cost of the algorithm is $$\Theta(n) + \Theta(n^2) + \Theta(n^3) = \Theta(n^3)$$
    }

    \section{Interconnection topologies}

    Up to this point, the routing problem has considered the network as a graph where \tbf{the structure is not known to the nodes}, and can change over time due to factors like \tit{faults} and \tit{variable traffic}. However, when the network represents an \tbf{interconnection topology}, such as one connecting processors, the structure of the network is known and remains fixed. This characteristic can be leveraged in the packet-routing algorithms.

    While the fixed nature of the network topology can be used to develop more efficient routing strategies, efficiency becomes a critical conecrn in interconnection topologies. As a result, solutions with stronger properties than basic shortest-path algorithms are required.

    There are many types of routing models. In this notes, the focus will be on the \href{https://en.wikipedia.org/wiki/Store_and_forward}{store-and-forward} model:

    \begin{itemize}
        \item aata is divided into \tit{discrete packets};
        \item each packet contains \tit{control information} (such as source, destination, and sequence data) and is treated as an independent unit that is forwarded from node to node through the network;
        \item packets may be temporarily stored in \tbf{buffer queues} at intermediate nodes if necessary, due to link congestion or busy channels;
        \item each node makes a \tbf{local routing decision} based on the packet's destination address and the chosen routing algorithm;
        \item during each step of the routing process, \tbf{a single packet can cross each edge};
        \item additionally, mechanisms for error detection and recovery may be employed to ensure reliable packet delivery, and flow control and congestion management may be applied to optimize network performance.
    \end{itemize}

    \subsection{Bufferfly network}

    \begin{frameddefn}{Bufferfly network}
        Let $n$ be an integer, and let $N := 2^n$; an \tbf{$n$-bufferfly network} is a \tit{layered graph} defined as follows:

        \begin{itemize}
            \item there are $n + 1$ layers of $N$ nodes each, for a total of $N(n + 1)$ nodes;
            \item each node is labeled with a pair $(w, i)$, where $i$ is the \tit{layer of the node}, and $w$ is an $n$-bit binary number that denotes the \tit{row of the node};
            \item there are $2Nn = 2 \cdot 2^n \cdot n = n2^{n + 1}$ edges;
            \item two nodes $(w, i)$ and $(w', i')$ are linked by an edge if and only if $i' = i + 1$ and either $w = w'$ (which is a \tit{straight edge}) or $w$ and $w'$ differ in only the $i$-th bit (which is a \tit{cross edge}).
        \end{itemize}
    \end{frameddefn}

    \begin{example}[Bufferfly network]
        TODO
    \end{example}

    The nodes of a butterfly are \tbf{crossbar switches}, which have two input and two output ports and can operate in two states, namely \tit{cross} and {bar} (shown below, respectively). \todo{mettere foto}

    It can be shown that each node (except those in the first and the last layers) has degree 4 \todo{prove it?}. Therefore, $4N$ additional nodes are typically added ($2N$ for the input, and $2N$ for the output) such that $\deg(u) = 4$ for each $u \in V(G)$ --- these nodes will not be considered in the networks analyzed in this notes. \todo{mettere foto}

    As a result, a butterfly network can be viewed as a \tit{switching network} that connects $2N$ input units to $2N$ ouptut units, through a layered structure divided into $\log N +1 = \log 2^n +1 = n + 1$ layers, each consisting of $N$ nodes.

    Moreover, butterfly networks have a recursive structure, which is highlighted in the following figure. Specifically, one $n$-dimensional butterfly contains two $(n - 1)$-dimensional butterfly networks as subgraphs. \todo{aggiungi foto}

    The topology of the butterfly network can be leveraged as stated in the following proposition.

    \begin{framedprop}{Greedy path}
        Given a pair of rows $w$ and $w'$, there exists a \tit{unique path of length $n$}, called \tbf{greedy path}, from node $(w, 0)$ to node $(w', n)$. This path passes through each layer exactly once, and it can be found through the following procedure:

        % \begin{algorithm}[H]
            \begin{algorithmic}[1]
                \Function{greedyPath}{$w$, $w'$}
                    \For{$i \in [1, n]$}
                        \If{$w_i == w'_i$}
                            \State Traverse a \tit{straight edge}
                        \Else
                            \State Traverse a \tit{cross edge}
                        \EndIf
                    \EndFor
                \EndFunction
            \end{algorithmic}
        % \end{algorithm}
    \end{framedprop}


\end{document}
